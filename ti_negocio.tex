\documentclass[review]{elsarticle}
\usepackage[utf8]{inputenc}
\usepackage[br]{nicealgo}
\usepackage{color}
\usepackage{natbib}
\usepackage[table,xcdraw]{xcolor}
\hyphenation{di-gi-tais}
\begin{document}

\begin{abstract}

Atualmente, todas as empresas, em um grau variável, precisam de um departamento que gerência a tecnologia da informação (TI), de forma a suprir a necessidade de crescimento e expectativas dos consumidores. O uso de setores de TI permite que negócios atendam as necessidades dos consumidores, aumentem a taxa de engajamento entre funcionários e acesso a informações com rapidez e flexibilidade de forma a responder a mudanças empresariais e desafios comerciais. Encontrar o balanço entre os setores de TI e empresarial de gerência requer especialistas que consigam visualizar como tais setores possam trabalhar juntos de forma. O alinhamento estratégico entre negócio e TI busca caminhos para que os lados possam trabalhar juntos de forma a maximizar o ganho.~\cite{Edmead2016,wikibit2018}


\end{abstract}

%titulo do seu trabalho
\title{Alinhamento Estratégico entre Negócio e TI}


\author{Nome e RA do Aluno I \\ Nome e RA do Aluno II \\ Prof. Clayton Pereira}% <-this % stops a space


\maketitle


\section{Introdução}
Segundo Sobrosa e Vanti (2009)~\cite{Sobrosa2009}, o alinhamento estratégico visa em fazer com que diversos setores de uma empresa se trabalhem de forma conjunta a aumentar o valor de suas atividades. Em uma empresa que possui um setor de TI, tal alinhamento precisa ser formulado de forma bem estruturada, já que os termos usados em um setor de TI nem sempre são claros para o resto da empresa, principalmente para a gerência. Assim, é preciso que a empresa seja bem estruturada e organizada de forma que os diferentes ramos possam se entender, e obter maior lucro. Em questão a estratégia, Sobrosa e Vanti avaliam que certas empresas preferem o planejamento enquanto outras preferem a concorrência no mercado, conforme visto em Table ~\ref{tabela_strat}. Esse trabalho visa apresentar as ideias de diversos artigos de forma a abordar o que é alinhamento estratégico entre negócio e TI.

\begin{table}[]
	\begin{tabular}{lll}
		\hline
		\rowcolor[HTML]{C0C0C0} 
		\textbf{(P) Estratégia}           & \textbf{Definição}                                                                                                    & \textbf{Características}                                                                                                                                                              \\ \hline
		\multicolumn{1}{|l|}{Plano}       & \multicolumn{1}{l|}{Curso ou ação, diretriz.}                                                                         & \multicolumn{1}{l|}{\begin{tabular}[c]{@{}l@{}}Preparadas previamente às ações.\\ Desenvolvidas consciente e deliberadamente.\end{tabular}}                                           \\ \hline
		\multicolumn{1}{|l|}{Pretexto}    & \multicolumn{1}{l|}{Manobra específica}                                                                               & \multicolumn{1}{l|}{\begin{tabular}[c]{@{}l@{}}Relacionada à estratégia como plano, com\\ intuito de ‘manobrar’ a concorrência.\end{tabular}}                                         \\ \hline
		\multicolumn{1}{|l|}{Padrão}      & \multicolumn{1}{l|}{\begin{tabular}[c]{@{}l@{}}Consistência de\\ comportamento.\end{tabular}}                         & \multicolumn{1}{l|}{\begin{tabular}[c]{@{}l@{}}Padrão relacionado à ação, com intenção.\\ Pode haver um plano implícito atrás do padrão.\end{tabular}}                                \\ \hline
		\multicolumn{1}{|l|}{Posição}     & \multicolumn{1}{l|}{\begin{tabular}[c]{@{}l@{}}Posição em relação a\\ uma referência.\end{tabular}}                   & \multicolumn{1}{l|}{\begin{tabular}[c]{@{}l@{}}Ponto de referência: ambiente, concorrente,\\ mercado.\\ Olhar para fora (posicionamento),\\ relacionando à organização.\end{tabular}} \\ \hline
		\multicolumn{1}{|l|}{Perspectiva} & \multicolumn{1}{l|}{\begin{tabular}[c]{@{}l@{}}Conceito da\\ organização,\\ visualizado\\ internamente.\end{tabular}} & \multicolumn{1}{l|}{\begin{tabular}[c]{@{}l@{}}Perspectiva compartilhada\\ Olhar para dentro (perspectiva), relacionando\\ à organização.\end{tabular}}                               \\ \hline
	\end{tabular}
	\caption{Definições de Estratégia, segundo os 5 Ps da Estratégia. \cite{Sobrosa2009}}
	
	\label{tabela_strat}
\end{table}

\section{Alinhamento Estratégico e TI}
Segundo Edmead (2009)~\cite{Edmead2016}, o lado gerente de uma empresa deveria ser o motivador dos planos estratégicos do lado de TI, mas isso não acontece porque a gerência fala uma linguagem totalmente diferente. Idealmente, o lado de TI consegue prover soluções táticas, que por sua vez é comandado pelo lado tático de negócios, de forma que o setor de TI consiga verificar se suas soluções estão gerando lucro. Uma das estratégias sugeridas seria o uso de BRMs (Bussiness Relationship Managers) que fariam a ponte entre o lado tático e o setor de TI: um BRM estratégico cuida da tradução entre os dois setores e o um BRM tático ajuda na implementação das decisões e infraestrutura. Além disso, eles atuam em conversas com pessoas chave dentro das empresas de forma que elas entendam quais as vantagens e façam total uso das implementações.

Burn e Setzo (2000)~\cite{BURN2000197} verificaram que empresas que tentaram realizar um alinhamento estratégico entre negócio e TI, somente 60{\%} tiveram 'grande sucesso' ou 'sucesso' na implementação, porém foi avaliado que o lado de tático de negócios que deve ser o motivador da implementação de estratégias, de forma que o alto escalão da empresa que deve alocar recursos de forma prioritária para os setores. Ainda assim, foi verificada a necessidade de um cargo de gerência na área de TI que possa se comunicar de forma eficiente com os outros setores. 

Coltman et. al.~\cite{Coltman15} analisa que na atualidade está cada vez mais difícil diferenciar os setores dentro de uma empresa de tecnologia, tal que sugere que o setor de TI precede a estratégia corporativa. Isso é verificado devido ao crescimento de negócios digitais: com a crescente necessidade de inovação e aplicação de novas tecnologias a necessidade de um alinhamento vem decaindo. Além disso, cada vez mais empresas vem mostrando o interesse a adquirir ou comprar sistemas já existentes do que desenvolver produtos proprietários. Ainda, Coltman et. al. mostram as seguintes questões de como poderia ser feito o alinhamento estratégico entre esses setores no futuro: 

\begin{itemize}
	\item Como diversos participantes em um ecossistema alinhar diferentes recursos de TI de forma a contribuírem de forma igual?
	\item Onde indivíduos com maior conhecimento sobre como conseguir o alinhamento estratégico em TI estarão localizados nesse ecossistema e como o seu conhecimento será distribuído entre ele?;
	\item Enquanto a tecnologia da informação vai ficando mais entrelaçada no sistema, como gerentes conseguirão melhorar o alinhamento com parceiros de forma que eles não se comportem oportunisticamente? No mesmo caso, como os efeitos do alinhamento para TI não se espalhem para outros participantes do ecossistema?
	\item Enquanto TI é associado com a agilidade em organizações, como esse alinhamento vai impactar a agilidade em setores únicos ou através do ecossistema?
\end{itemize}

\section{Conclusão}


\bibliographystyle{abntex2-num}
\bibliography{refs}
\end{document}


